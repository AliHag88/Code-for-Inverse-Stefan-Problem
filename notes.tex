\documentclass[letterpaper, 10pt]{amsart}

\usepackage[T1]{fontenc}
\usepackage{graphicx, baskervald}
\usepackage{geometry}

\usepackage{matlab-prettifier}

\usepackage[inline]{showlabels}

\let\institute\address%
\newtheorem{theorem}{Theorem}
\newtheorem{lemma}{Lemma}
\newtheorem{corollary}{Corollary}
\theoremstyle{definition}
\newtheorem{definition}{Definition}
\theoremstyle{remark}
\newtheorem{remark}{Remark}

\newcommand{\D}[2]{\frac{\partial{} #1}{\partial{} #2}}
\newcommand{\dD}[2]{\frac{d #1}{d #2}}
\newcommand{\bk}[1]{\left\{#1\right\}}
\newcommand{\norm}[1]{\left\Vert #1\right\Vert}%chktex 1 ok (math-mode)
\newcommand{\lnorm}[1]{\left\vert #1\right\vert}%chktex 1 ok (math-mode)
\DeclareMathOperator{\arccoth}{\text{arccoth}}

\begin{document}

\title{Implementation of ISP}
\author{Ali Haqverdiyev \and Jonathan Goldfarb}
\institute{Department of Mathematical Sciences\\Florida Institute of
    Technology\\Melbourne, FL 32901}
\subjclass[2010]{TBD}
\date{\today}

\begin{abstract}
    TBD

    \smallskip%
    \noindent\textbf{\keywordsname.} TBD
\end{abstract}
\maketitle

\tableofcontents{}

\section{Introduction}\label{sec:introduction}

\subsection{Inverse Stefan Problem}\label{sec:inverse-stefan-problem}
Consider the general one-phase Stefan problem:
\begin{gather}
  Lu \equiv {(a(t) b(x) u_x)}_x - u_{t} = f,~\text{in}~\Omega \label{eq:intro-pde}
  \\
  u(x,0) = \phi (x),~0 \leq x \leq s(0)=s_0 \label{eq:intro-init}
  \\
  a(t) b(0) u_x (0,t) = g(t),~0 \leq t \leq T \label{eq:pde-bound}
  \\
  a(t) b(s(t)) u_x (s(t),t)
  + \gamma (s(t),t) s'(t)
  = \chi (s(t),t),~0 \leq t \leq T \label{eq:pde-stefan}
  \\
  u(s(t),t) = \mu (t),~0 \leq t \leq T, \label{eq:pde-freebound}
  \intertext{where}
  \Omega = \{(x,t): 0 < x < s(t),~0 < t \leq T\}\label{eq:pde-domain-defn}
\end{gather}
where $a$, $b$, $f$, $\phi$, $g$, $\gamma$, $\chi$, $\mu$ are given functions.
Assume now that $a$ is unknown; in order to find $a$, along with $u$ and $s$, we must have additional information.
Assume that we are able to measure the temperature on our domain and the position of the free boundary at the final moment $T$.
\begin{equation}
    u(x,T) = w(x),~ 0 \leq x \leq s(T) = s_*.\label{eq:pde-finaltemp}
\end{equation}
Under these conditions, we are required to solve an \emph{inverse} Stefan problem (ISP): find a tuple
\def\controlvarsWithArgs{s(t), a(t)}
\def\controlvars{s, a}
\def\controlvarsWithDelta{{\Delta s}, {\Delta a}}%chktex 1 ok (math-mode)
\[
    \bk{
        u(x,t), \controlvarsWithArgs
    }
\]
that satisfy conditions~\eqref{eq:intro-pde}--\eqref{eq:pde-finaltemp}.
ISP is not well posed in the sense of Hadamard: the solution may not exist; if it exists, it may not be unique, and in general it does not exhibit continuous dependence on the data.
The main methods available for ISP are based on a variational formulation, Frechet differentiability, and iterative gradient methods.
We cite recent papers~\cite{abdulla13,abdulla15} and the monograph~\cite{goldman97} for a list of references.
The established variational methods in earlier works fail in general to address two issues:
\begin{itemize}
    \item The solution of ISP does not depend continuously on the phase transition temperature $\mu(t)$ from~\eqref{eq:pde-freebound}.
    A small perturbation of the phase transition temperature may imply significant change of the solution to the inverse Stefan problem.
    \item In the existing formulation, at each step of the iterative method a Stefan problem must be solved (that is, for instance, the unknown heat flux $g$ is given, and the corresponding $u$ and $s$ are calculated) which incurs a high computational cost.
\end{itemize}

A new method developed in~\cite{abdulla13,abdulla15} addresses both issues with a new variational formulation.
The key insight is that the free boundary problem has a similar nature to an inverse problem, so putting them into the same framework gives a conceptually clear formulation of the problem; proving existence, approximation, and differentiability is a resulting challenge.
Existence of the optimal control and the convergence of the sequence of discrete
optimal control problems to the continuous optimal control problem was proved
in~\cite{abdulla13,abdulla15}.
In~\cite{abdulla16,abdulla17}, Frechet differentiability of the new variational
formulation was developed, and a full result showing Frechet differentiability
and the form of the Frechet differential
with respect to the free boundary, sources, and coefficients were proven.
Our goal in this work is to document the implementation in MATLAB using the built-in PDEPE solver.


\section{Forward Problem for ISP}\label{sec:forward-problem}
We will consider the problem~\eqref{eq:intro-pde}--\eqref{eq:pde-stefan} where $(\controlvars{}) \in V_R$ is fixed, and
\begin{align}
  V_R & =\Big\{
        v=(\controlvars{}) \in H, \lnorm{v}_H \leq R;
        ~s(0) = s_0,
        ~g(0) = a(0)b(0) \phi'(0),
        ~a_0 \leq a(t)\nonumber
  \\
      & \qquad
        ~\chi(s_{0},0) = \phi'(s_{0})a(0)b(s_0) + \gamma(s_0, 0)s'(0),
        ~0 < \delta \leq s(t)
        \Big\},\label{eq:control-set}
\end{align}
where $\beta_0, \beta_1, \beta_2 \geq 0$ and $a_0, \delta, R > 0$ are given, and
\def\acontrolspace{W_2^1}
\def\scontrolspace{W_2^2}
\begin{gather*}
  H := \acontrolspace(0,T)
  \times \scontrolspace(0,T)
    \\
    \norm{v}_H := \max\Big(
    \norm{a}_{\acontrolspace(0,T)},
    \norm{s}_{\scontrolspace(0,T)}
    \Big)
  \end{gather*}
  Define
  \let\l\ell%
\[
  D := \bk{(x,t) : 0\leq x\leq \l,~ 0\leq t\leq T},
\]
where $l = l(R) > 0$ is chosen such that for any control $v\in V_R$, its component
$s$ satisfies $s(t)\leq l$.
For a given control vector $v = (\controlvars{}) \in V_R$ transform the domain $\Omega$ to the cylindrical domain $Q_T$
by the change of variables $y = x / s(t)$.
Let $d = d(x, t)$, $(x, t) \in \Omega$ stand for any of $u$, $b$, $f$, $\gamma$, $\chi$, define the function $\tilde{d}$ by
\begin{gather*}
  \tilde{d}(x,t) = d\big(x s(t), t\big),~
  \tilde{b}(x) = b(x s(t)),~\text{and}~
  \tilde{\phi}(x) = \phi\big( x s_0\big)
\end{gather*}
\def\utilde{\tilde{u}}
The transformed function $\utilde$ is a \emph{pointwise a.e.} solution of the Neumann problem
\begin{gather}
  \frac{a}{s^2}\big(\tilde{b} \utilde_y\big)_y + \frac{y s'}{s} \utilde_y - \utilde_{t} = \tilde{f}, ~\text{in}~Q_T\label{eq:tform-pde}
  \\
  \utilde(x,0) = \tilde{\phi}(x), ~0 \leq x \leq 1 \label{eq:tform-iv}
  \\
  a(t) b(0) \utilde_y(0, t) = g(t)s(t), ~0 \leq t \leq T   \label{eq:tform-lbdy}
  \\
  a(t) b(1) \utilde_y(1, t) = \tilde{\chi}(1, t) s(t) - \tilde{\gamma}(1,t)s'(t)s(t), ~0 \leq t \leq T \label{eq:tform-rbdy}
\end{gather}


\subsection{MATLAB Implementation of Forward Problem}
The built-in MATLAB PDE solver, \verb+pdepe+, solves parabolic-elliptic problems in one space dimension.
In particular, it uses a method-of-lines technique as a differential-algebraic equation solver applied to problems of the form
\begin{gather}
  c\left(x,t,u,\D{u}{x}\right) \D{u}{t}
  = x^{-m} \D{}{x} \left( x^m F\left(x,t,u,\D{u}{x} \right)\right)
  + d\left( x,t,u,\D{u}{x}\right)\label{eq:matlab-pde}
  \intertext{for}
  a \leq x \leq b,\quad t_0 \leq t \leq t_f
  \intertext{where $m=0,1,2$ is fixed, $c$ is a diagonal matrix of size $n\times n$, where there are $n$ components in $u$.
    There must be at least one parabolic equation, which corresponds to the condition of at least one component of $c$ being positive.
    The solution components satisfy}
  u(x,t_0) = u_0(x)\label{eq:matlab-ic}
  \intertext{and boundary conditions of the form}
  p(x,t,u)
  + q(x,t) F\left(x,t,u,\D{u}{x} \right)\Big\vert_{x=a,b}
  = 0\label{eq:matlab-bc}
\end{gather}
In particular, the function $F$ appearing in the boundary condition is the same as the flux term in the PDE.\@
In the MATLAB implementation, the method \verb+Forward+ is intended to produce a function $u(x,t)$ by computing $\utilde(y,t)$ and returning $u(x,t) = \utilde(x/s(t),t)$.
The subfunction \verb+pdeSolver+ handles the conversion of problems of the form~\eqref{eq:tform-pde}--\eqref{eq:tform-rbdy} to the form that the MATLAB solver requires.

We first write system~\eqref{eq:tform-pde}--\eqref{eq:tform-rbdy} in the form
\begin{gather}
  % a \big(\tilde{b} \utilde_y\big)_y
  % + s y s' \utilde_y
  % - s^2\utilde_{t}
  % = s^2 \tilde{f}, ~\text{in}~Q_T
  % \\
  % \intertext{Or, multiplying the first equation through by $-1$,}
  s^2 \utilde_t = a \big(\tilde{b} \utilde_y\big)_y
  + s y s' \utilde_y
  - s^2 \tilde{f}, ~\text{in}~Q_T=:(0,1) \times (0,T)
  \\
  \utilde(x,0) = \tilde{\phi}(x), ~0 \leq x \leq 1
  \\
  a(t) b(0) \utilde_y(0, t) = g(t)s(t), ~0 \leq t \leq T
  \\
  a(t) \tilde{b}(1) \utilde_y(1, t)
  = \tilde{\chi}(1, t) s(t)
  - \tilde{\gamma}(1,t)s'(t)s(t), ~0 \leq t \leq T
\end{gather}

Evidently, we must take $a=0$, $b=1$, $t_0=0$, $t_f = T$, and $m=0$; % , giving
% \[
%   c\left(x,t,u,\D{u}{x}\right) \D{u}{t}
%   = \D{}{x} F\left(x,t,u,\D{u}{x} \right)
%     + d\left( x,t,u,\D{u}{x}\right)
% \]
% By
then, identifying the diffusion coefficient between~\eqref{eq:tform-pde} and~\eqref{eq:matlab-pde}, we observe that
\[
  F(x, t, u, r) := a(t) \tilde{b}(x) r
\]
and by considering the time derivative, evidently
\[
  c(x, t, u, r) := s^2(t)
\]
and lastly,
\[
  d(x, t, u, r) := s(t) \big(s'(t) x r - s(t) \tilde{f}(x,t)\big)
\]
The initial data should be taken as $u_0(x) = \tilde{\phi}(x) = \phi(x s(t))$, and the boundary conditions take the form
\begin{gather*}
  a(t)\tilde{b}(0) \utilde_y(0,t) \equiv F(0, t, \utilde, \utilde_y)
  = p(0,t,\utilde):=g(t) s(t),
  \\
  q(0,t)=1
  \\
  a(t)\tilde{b}(1) \utilde_y(1,t) \equiv F(1, t, \utilde, \utilde_y)
  = p(1,t,\utilde)
  := s(t)\left(
    \tilde{\chi}(1,t) - \tilde{\gamma}(1,t) s'(t)
    \right)
\end{gather*}

\subsection{Formulation of Reduced Model for Forward Problem}
Consider the problem~\eqref{eq:intro-pde}--\eqref{eq:pde-stefan} with $\gamma\equiv 1$, $\chi \equiv 0$, and $b \equiv 1$:
\begin{gather}
  Lu \equiv a(t) u_{xx} - u_{t} = f,~\text{in}~\Omega \label{eq:reduced-pde}
  \\
  u(x,0) = \phi (x),~0 \leq x \leq s(0)=s_0 \label{eq:reduced-init}
  \\
  a(t) u_x (0,t) = g(t),~0 \leq t \leq T \label{eq:reduced-lbc}
  \\
  a(t) u_x (s(t),t)
  +  s'(t)
  = 0,~0 \leq t \leq T \label{eq:reduced-stefan}
\end{gather}

The corresponding problem~\eqref{eq:tform-pde}--\eqref{eq:tform-rbdy} reduces to to give
\begin{gather}
  % a \big(\tilde{b} \utilde_y\big)_y
  % + s y s' \utilde_y
  % - s^2\utilde_{t}
  % = s^2 \tilde{f}, ~\text{in}~Q_T
  % \\
  % \intertext{Or, multiplying the first equation through by $-1$,}
  s^2 \utilde_t = a \big( \utilde_y\big)_y
  + s y s' \utilde_y
  - s^2 \tilde{f}, ~\text{in}~Q_T
  \\
  \utilde(y,0) = \tilde{\phi}(y), ~0 \leq x \leq 1
  \\
  a(t) \utilde_y(0, t) = g(t)s(t), ~0 \leq t \leq T
  \\
  a(t) \utilde_y(1, t)
  = - s'(t)s(t), ~0 \leq t \leq T
\end{gather}
and the mapping from the notation in our paper to the notation preferred by MATLAB becomes
\begin{gather*}
  F(x, t, u, r) := a(t) r
  \\
  c(x, t, u, r) := s^2(t)
  \\
  d(x, t, u, r) := s(t) \big(s'(t) x r - s(t)\tilde{f}(x,t)\big)
\intertext{The initial data should be taken as $u_0(x) = \tilde{\phi}(x) = \phi(x s(t))$, and the boundary conditions take the form}
  a(t) \utilde_y(0,t) \equiv F(0, t, \utilde, \utilde_y)
  = p(0,t,\utilde)
  := g(t) s(t),
  \\
  a(t) \utilde_y(1,t) \equiv F(1, t, \utilde, \utilde_y)
  = p(1,t,\utilde)
  := -s(t) s'(t)
  \intertext{and}
  q(x,t) \equiv 1
\end{gather*}

Under the additional assumption that the sources $f \equiv 0$, the mapping simplifies further to
\begin{gather*}
  F(x, t, u, r) := a(t) r
  \\
  c(x, t, u, r) := s^2(t)
  \\
  d(x, t, u, r) := s(t) s'(t) x r
\end{gather*}
The boundary conditions take the same form with the updated definition of $F$.
This is the model implemented in \verb+Forward.m+ (specifically, in the \verb+pdeSolver+ subfunction), included in Section~\ref{sec:code-listing-forward}.
The model problem currently implemented in \verb+test_Forward.m+, included in Section~\ref{sec:code-listing-test-forward} can be found in Section~\ref{sec:model-problem-1}.

\section{Optimal Control Problem and Gradient-Based Approach}
Consider the minimization of the functional
\def\J{\mathcal{J}}
\begin{equation}
  \J(v)
  = \beta_0 \int_0^{s(T)} \lnorm{u(x, T; v) - w(x)}^2\,dx
  + \beta_1 \int_0^T \lnorm{u(s(t), t; v) - \mu(t)}^2\,dt
  + \beta_2 \lnorm{s(T) - s_*}^2\label{eq:functional}
\end{equation}
on the control set~\eqref{eq:control-set}.
\def\I{\mathcal{I}}
The formulated optimal control problem~\eqref{eq:control-set},~\eqref{eq:tform-pde}--\eqref{eq:tform-rbdy},~\eqref{eq:functional} will be called Problem $\I$.

\begin{definition}\label{defn:adjoint}
  For given $v$ and $u = u(x, t; v)$, $\psi$ is a solution to the adjoint problem if
  \begin{gather}
    L^{*} \psi := \big(a b \psi_x\big)_x + \psi_t = 0,\quad\text{in}~\Omega \label{eq:adj-pde}
    \\
    \psi(x, T) = 2\beta_0(u(x, T) - w(x)),~0 \leq x \leq s(T) \label{eq:adj-finalmoment}
    \\
    b(0)a(t)\psi_x(0, t) =0,~0 \leq t \leq T \label{eq:adj-robin-fixed}
    \\
    \Big[a b \psi_x - s'\psi - 2\beta_1(u - \mu)\Big]_{x=s(t)}=0, ~0 \leq t \leq T \label{eq:adj-robin-free}
  \end{gather}
\end{definition}


\begin{theorem}\label{thm:gradient-result}
  Problem $\I$ has a solution, and the functional $\J(v)$ is differentiable in the sense of Frechet, and the first variation is
  \begin{gather}
    \left\langle \J'(v),\Delta v \right\rangle_H
    = \int_0^T \Big[2\beta_1\big(u - \mu\big)u_x + \psi \big(\chi_x - \gamma_x s'  - a (b u_x)_x \big)\Big]_{x=s(t)} {\Delta s}(t)\,dt \nonumber
    \\
    +
    \left[\beta_0\lnorm{u(s(T),T) - w(s(T))}^2 + 2 \beta_2 (s(T) - s_*)\right] {\Delta s}(T) 
    - \int_0^T \big[\psi \gamma\big]_{x=s(t)} {\Delta s}'(t) \,dt \nonumber
    \\
    +\int_0^T {\Delta a} \left[ \int_0^{s(t)} {(b u_x)}_x \psi \,dx - [b u_x\psi]_{x=s(t)} - [b u_x \psi]_{x=0}\right] \,dt\label{eq:gradient-full}
  \end{gather}
  where $\J'(v) \in H'$ is the Frechet derivative, $\langle \cdot,\cdot \rangle_H$ is a pairing between $H$ and its dual $H'$, $\psi$ is a solution to the adjoint problem in the sense of definition~\ref{defn:adjoint}, and $\delta v = (\controlvarsWithDelta{})$ is a variation of the control vector $v \in V_R$ such that $v + \delta v \in V_R$.
\end{theorem}

\subsection{MATLAB Implementation of Adjoint Problem}

To implement the adjoint problem, we first change variables in eqs.~\eqref{eq:adj-pde}--\eqref{eq:adj-robin-free} as $x \mapsto x/s(t)$, $t \mapsto T-t$ to derive
% \begin{gather}
%   L^{*} \psi
%   % = \frac{1}{s^2(t)}\big(a(t) b(ys(t)) \tilde{\psi}_y\big)_y
%   % - y s'(t)/s(t) \tilde{\psi}_y
%   % + \tilde{\psi}_t = 0
%   %% Which is
%   a(t) \big(b(ys(t)) \tilde{\psi}_y\big)_y
%   - s(t) y s'(t) \tilde{\psi}_y
%   + s^2(t) \tilde{\psi}_t = 0
%   ,\quad\text{in}~Q_T:=(0,1)\times (0,T)
%   \\
%   \tilde{\psi}(y, T) = 2\beta_0(u(y s(T), T) - w(y s(T))),~0 \leq y \leq 1
%   \\
%   b(0)a(t)\tilde{\psi}_y(0, t) =0,~0 \leq t \leq T
%   \\
%   \frac{a(t) b(s(t))}{s(t)} \tilde{\psi}_y(1,t) = s'(t)\tilde{\psi}(1,t) + 2\beta_1(u(s(t),t) - \mu(t))=0, ~0 \leq t \leq T
% \end{gather}
% Replace $t$ with $-t$ to derive
% \begin{gather}
%   a(-t) \big(b(ys(-t)) \tilde{\psi}_y\big)_y
%   + s(-t) y s'(-t) \tilde{\psi}_y
%   - s^2(-t) \tilde{\psi}_t = 0
%   ,\quad\text{in}~Q_T:=(0,1)\times (0,-T)
%   \\
%   \tilde{\psi}(y, -T) = 2\beta_0(u(y s(T), T) - w(y s(T))),~0 \leq y \leq 1
%   \\
%   b(0)a(-t)\tilde{\psi}_y(0, -t) =0,~-T \leq t \leq 0
%   \\
%   a(-t) b(s(-t)) \tilde{\psi}_y(1,t)
%   = -s(T-t) s'(T-t)\tilde{\psi}(1,t)
%   + 2 \beta_1 s(T-t) (u(s(T-t),t) - \mu(T-t))=0, ~-T \leq t \leq 0
% \end{gather}
% Lastly, replace $t$ with $t+T$ to derive
\begin{gather}
  a(T-t) \big(b(ys(T-t)) \tilde{\psi}_y\big)_y
  + s(T-t) y s'(T-t) \tilde{\psi}_y
  - s^2(T-t) \tilde{\psi}_t = 0
  ,\quad\text{in}~Q_T:=(0,1)\times (0,T)
  \\
  \tilde{\psi}(y, 0) = 2\beta_0(u(y s(T), T) - w(y s(T))),~0 \leq y \leq 1
  \\
  b(0)a(T-t)\tilde{\psi}_y(0, T-t) =0,~0 \leq t \leq T
  \\
  a(T-t) b(s(T-t)) \tilde{\psi}_y(1,t)
  = -s(T-t) s'(T-t)\tilde{\psi}(1,t)
  + 2 \beta_1 s(T-t) (u(s(T-t),t) - \mu(T-t))=0, ~0 \leq t \leq T
\end{gather}
Comparing with the notation of eqs.~\eqref{eq:matlab-pde}--\eqref{eq:matlab-bc}, we see that once again we must take $a=0$, $b=1$, $t_0=0$, $t_f = T$, and $m=0$; hence~\eqref{eq:matlab-pde} has the form
\[
  c\left(x,t,u,\D{u}{x}\right) \D{u}{t}
  = \D{}{x} F\left(x,t,u,\D{u}{x} \right)
  + d\left( x,t,u,\D{u}{x}\right)
\]
so
\begin{gather*}
  F(x, t, u, r)
  := a(T-t) b(ys(T-t)) r
  \\
  c(x, t, u, r) := s^2(T-t)
  \\
  d(x, t, u, r) := x s(T-t) s'(T-t) r
\end{gather*}
The initial condition~\eqref{eq:matlab-ic} has the form
\[
  \tilde{\psi}(x,0)= 2\beta_0(u(x s(T), T) - w(x s(T)))
\]
The boundary conditions have the form
\begin{gather*}
  b(0)a(T-t)\psi_y(0, t)
  \equiv F(0, t, \psi, \psi_y(0,t)) = 0 \equiv p(0,t,\psi)
  \\
  a(T-t) b(s(T-t)) \tilde{\psi}_y(1,t)
  \equiv F(1, t, \psi, \psi_y(1,t))
  = p\big(1, t, \psi(1, t)\big),
  \\
  p\big(1,t,r\big)
  :=-s(T-t) s'(T-t)r
  + 2 \beta_1 s(T-t) (u(s(T-t),t) - \mu(T-t))
\end{gather*}

\subsection{Formulation of Reduced Model for Adjoint Problem}
In the reduced model, we assume $\gamma \equiv $, $\chi \equiv 0$, and $b\equiv 1$, so the adjoint problem becomes
\begin{gather}
  \big(a \psi_x\big)_x + \psi_t = 0,\quad\text{in}~\Omega
  \\
  \psi(x, T) = 2\beta_0(u(x, T) - w(x)),~0 \leq x \leq s(T)
  \\
  a(t)\psi_x(0, t) =0,~0 \leq t \leq T
  \\
  \Big[a \psi_x - s'\psi - 2\beta_1(u - \mu)\Big]_{x=s(t)}=0, ~0 \leq t \leq T
\end{gather}
and the transformed problem is
\begin{gather}
  a(T-t) \tilde{\psi}_{yy}
  + s(T-t) y s'(T-t) \tilde{\psi}_y
  - s^2(T-t) \tilde{\psi}_t = 0
  ,\quad\text{in}~Q_T:=(0,1)\times (0,T)
  \\
  \tilde{\psi}(y, 0) = 2\beta_0(u(y s(T), T) - w(y s(T))),~0 \leq y \leq 1
  \\
  a(T-t)\tilde{\psi}_y(0, T-t) =0,~0 \leq t \leq T
  \\
  a(T-t) \tilde{\psi}_y(1,t)
  = -s(T-t) s'(T-t)\tilde{\psi}(1,t)
  + 2 \beta_1 s(T-t) (u(s(T-t),t) - \mu(T-t))=0, ~0 \leq t \leq T
\end{gather}

Therefore, corresponding parameter functions for the MATLAB interface take the form
\begin{gather*}
  F(x, t, u, r)
  := a(T-t) r
  \\
  c(x, t, u, r)
  := s^2(T-t)
  \\
  d(x, t, u, r)
  := x s(T-t) s'(T-t) r
\end{gather*}
and the boundary conditions remain the same after updating the definition of $F$.

This model is implemented in \verb+Adjoint.m+ (specifically, in the \verb+pdeSolver+ subfunction), included in Section~\ref{sec:code-listing-adjoint}.
The development of a reasonable formulation for a test problem for the adjoint solver is a work-in-progress; for now, the same problem is used in \verb+test_Forward.m+ and in \verb+test_Adjoint.m+ (taking the appropriate boundary measurements; see Section~\ref{sec:code-listing-test-adjoint}) and we simply check that the adjoint problem vanishes as the measurement error vanishes.

\subsection{Common PDE Solver for Forward and Adjoint Problem}
Comparing the parameter functions for the reduced Forward problem,
\begin{gather*}
  F(x, t, u, r) := a(t) r
  \\
  c(x, t, u, r) := s^2(t)
  \\
  d(x, t, u, r) := s(t) s'(t) x r
  \\
  q(x,t) :=1
  \\
  p(0, t, r) := g(t) s(t)
  \\
  p(1, t, r) := -s(t) s'(t)
\end{gather*}
and the reduced Adjoint problem,
\begin{gather*}
  F(x, t, u, r)
  := a(T-t) r
  \\
  c(x, t, u, r)
  := s^2(T-t)
  \\
  d(x, t, u, r)
  := y s(T-t) s'(T-t) r
  \\
  q(x,t) :=1
  \\
  p(0, t, r) := 0
  \\
  p(1, t, r) := -s(T-t) s'(T-t)r
  + 2 \beta_1 s(T-t) (u(s(T-t),t) - \mu(T-t))
\end{gather*}
we see that both problems have the same fundamental structure; exploiting this to reduce code duplication is a future project.

\subsection{Implementation of Gradient Formulae in MATLAB}
The last step in the implementation is to write the gradient update formula in MATLAB notation.
For the gradient with respect to $x=s(t)$, we have
\begin{gather*}
  \int_0^T \Big[2\beta_1\big(u - \mu\big)u_x + \psi \big(\chi_x - \gamma_x s'  - a (b u_x)_x \big)\Big]_{x=s(t)} {\Delta s}(t)\,dt \nonumber
  \\
  +
  \left[\beta_0\lnorm{u(s(T),T) - w(s(T))}^2 + 2 \beta_2 (s(T) - s_*)\right] {\Delta s}(T) 
  - \int_0^T \big[\psi \gamma\big]_{x=s(t)} {\Delta s}'(t) \,dt
\end{gather*}
Under the assumptions used to formulate ``reduced'' models above ($\gamma \equiv 1$, $\chi \equiv 0$, $b\equiv 1$,) we have
\begin{gather*}
  \int_0^T \Big[2\beta_1\big(u - \mu\big)u_x - \psi  a u_{xx} \Big]_{x=s(t)} {\Delta s}(t)\,dt \nonumber
  \\
  +
  \left[\beta_0\lnorm{u(s(T),T) - w(s(T))}^2 + 2 \beta_2 (s(T) - s_*)\right] {\Delta s}(T) 
  - \int_0^T \psi(s(t),t) {\Delta s}'(t) \,dt
\end{gather*}
Pursuing integration by parts with respect to time, it follows that the corresponding gradient term is
\begin{gather*}
  % \int_0^T \Big[2\beta_1\big(u - \mu\big)u_x - \psi  a u_{xx} \Big]_{x=s(t)} {\Delta s}(t)\,dt \nonumber
  % \\
  % +
  % \left[\beta_0\lnorm{u(s(T),T) - w(s(T))}^2 + 2 \beta_2 (s(T) - s_*)\right] {\Delta s}(T) 
  % + \int_0^T \D{}{t}\psi(s(t),t) {\Delta s}(t) \,dt
  % \\
  % - \psi(s(T),T) {\Delta s}(T)
  % + \psi(s(0),0) {\Delta s}(0)
  % \intertext{and hence}
  % \int_0^T \Big[2\beta_1\big(u - \mu\big)u_x - \psi  a u_{xx} \Big]_{x=s(t)} {\Delta s}(t)\,dt \nonumber
  % \\
  % +
  % \left[\beta_0\lnorm{u(s(T),T) - w(s(T))}^2 + 2 \beta_2 (s(T) - s_*) - \psi(s(T),T)\right] {\Delta s}(T) 
  % + \int_0^T \D{}{t}\psi(s(t),t) {\Delta s}(t) \,dt
  % \\
  % =
  \int_0^T \Big[2\beta_1\big(u - \mu\big)u_x - \psi  a u_{xx}
  + \psi_x(s(t),t) s'(t) + \psi_t(s(t),t)\Big]_{x=s(t)} {\Delta s}(t)\,dt \nonumber
  \\
  +
  \left[\beta_0\lnorm{u(s(T),T) - w(s(T))}^2 + 2 \beta_2 (s(T) - s_*) - \psi(s(T),T)\right] {\Delta s}(T) 
\end{gather*}
This is the routine implemented in \verb+grad_s.m+, included in Section~\ref{sec:code-listing-grad-s}.

\section{Model Problem \#1}\label{sec:model-problem-1}
As a model problem, we will take~\eqref{eq:intro-pde}--\eqref{eq:pde-stefan} with $b\equiv 1$, $\chi \equiv 0$, and $\gamma \equiv 1$:
\begin{gather}
  Lu \equiv {(a(t) u_x)}_x - u_{t} = f,~\text{in}~\Omega
  \\
  u(x,0) = \phi (x),~0 \leq x \leq s(0) = s_0
  \\
  a(t) u_x (0,t) = g(t),~0 \leq t \leq T
  \\
  a(t) u_x (s(t),t) + s'(t) = 0,~0 \leq t \leq
  T
\end{gather}
In this case, the adjoint problem~\eqref{eq:adj-pde}--\eqref{eq:adj-robin-free} then takes the form
\begin{gather}
  L^{*} \psi := \big(a \psi_x\big)_x + \psi_t = 0,\quad\text{in}~\Omega
  \\
  \psi(x, T) = 2\beta_0(u(x, T) - w(x)),~0 \leq x \leq s(T)
  \\
  a(t)\psi_x(0, t) =0,~0 \leq t \leq T
  \\
  \Big[a \psi_x - s'\psi - 2\beta_1(u - \mu)\Big]_{x=s(t)}=0, ~0 \leq t \leq T
\end{gather}
and the gradient is
\begin{gather}
  \left\langle \J'(v),\Delta v \right\rangle_H
  = \int_0^T \Big[2\beta_1\big(u - \mu\big)u_x - a \psi u_{xx} \Big]_{x=s(t)} {\Delta s}(t)\,dt \nonumber
  \\
  +
  \left[\beta_0\lnorm{u(s(T),T) - w(s(T))}^2 + 2 \beta_2 (s(T) - s_*)\right] {\Delta s}(T)
  - \int_0^T \psi \big\vert_{x=s(t)} {\Delta s}'(t) \,dt\nonumber
  \\
  + \int_0^T {\Delta a} \left[
    \int_0^{s(t)}  u_{xx} \psi \,dx
    - [u_x \psi]_{x=0}
    - [u_x\psi]_{x=s(t)}
  \right]\,dt
\end{gather}

% One such problem can be found in the text by Tikhonov \& Samarskii~\cite[Ch.\ II, App.\ IV, pp.\ 283--288]{tikhonov63}.
The model problem originates in the following: find \(\big(u, \xi{}\big)\) satisfying
\begin{gather}
  \D{u_1}{t} - k_1 \D{^2 u_1}{x^2} = 0,~0<x<\xi(t),~t>0\label{eq:tikhonov-samarskii-pde-1}
  \\
  \D{u_2}{t} - k_2 \D{^2 u_2}{x^2} = 0,~\xi(t)<x<\infty,~t>0\label{eq:tikhonov-samarskii-pde-2}
  \\
  u_1(0,t) = c_1,~t>0
  \\
  u_2(x,0) = c_2,~x>\xi(0)
  \\
  u_1(x,0) = c_1,~x<\xi(0) % Note: this condition is added since $\xi$ is
  % unknown a-priori.
  \\
  u_1(\xi(t),t) = u_2(\xi(t),t)=0,~t>0
  \\
  k_1 \D{u_1}{x}(\xi(t),t) - k_2 \D{u_2}{x}(\xi(t),t) = \gamma \dD{\xi}{t}(t)\label{eq:tikhonov-samarskii-stefancond}
\end{gather}
where \(k_1, k_2, \gamma >0\) are given, and without loss of generality \(c_1<0\), \(c_2\geq 0\).
Direct calculation verifies that the exact solution to~\eqref{eq:tikhonov-samarskii-pde-1}--\eqref{eq:tikhonov-samarskii-stefancond} is
\begin{gather}
  u_1(x,t)=c_1 + B_1 \mathrm{erf}\left( \frac{x}{\sqrt{4 k_1 t}}\right)
  \\
  u_2(x,t)=A_2 + B_2 \mathrm{erf}\left( \frac{x}{\sqrt{4 k_2 t}}\right)
  \\
  \xi(t) = \alpha \sqrt{t}
  \intertext{where}
  \mathrm{erf}(x)=\frac{2}{\sqrt{\pi}} \int_0^x e^{-z^2} \,dz,
  \\
  B_1 =-\frac{c_1}{\mathrm{erf}\left(\alpha/\sqrt{4 k_1}\right)},\quad
  B_2 = \frac{c_2}{1-\mathrm{erf}\left(\alpha/\sqrt{4 k_2}\right)}
  \\
  A_2 =-\mathrm{erf}\left(\alpha/\sqrt{4 k_2}\right) B_2
  \intertext{and \(\alpha{}\) is a solution of the transcendental equation}
  % %% Directly translated
  %\frac{k_1 c_1 e^{-\frac{\alpha^2}{4 k_1}}}{\sqrt{k_1} \Phi\left(\alpha/\sqrt{4 k_1}\right)}
  %+ \frac{k_2 c_2 e^{-\frac{\alpha^2}{4 k_2}}}{\sqrt{k_2}\left[1-\Phi\left(\alpha/\sqrt{4 k_2}\right) \right]}
  %= -\frac{\gamma \sqrt{\pi}}{2} \alpha
  % %% But some cancellation happens since we chose particular values for
  % a_1/a_2 from reference.
  \frac{\sqrt{k_1} c_1 e^{-\frac{\alpha^2}{4 k_1}}}{ \mathrm{erf}\left(\alpha/\sqrt{4 k_1}\right)}
  + \frac{\sqrt{k_2} c_2 e^{-\frac{\alpha^2}{4 k_2}}}{\left[1-\mathrm{erf}\left(\alpha/\sqrt{4 k_2}\right) \right]}
  = -\frac{\gamma \sqrt{\pi}}{2} \alpha
\end{gather}
The equation above has a unique solution \(\alpha>0\).

To write the problem as a one-phase Stefan problem, we simply set $c_2=0$, so $u_2 \equiv 0$, and the function $u=u_1$ satisfies
\begin{gather}
  \D{u}{t} - k_1 \D{^2 u}{x^2} = 0,~0<x<\xi(t),~t>0
  \\
  u(0,t) = c_1,~t>0
  \\
  u(x,0) = c_1,~x<\xi(0) % Note: this condition is added since $\xi$ is
  % unknown a-priori.
  \\
  u(\xi(t),t) = 0,~t>0
  \\
  k_1 \D{u}{x}(\xi(t),t) = \gamma \dD{\xi}{t}(t)
\end{gather}
where \(k_1, k_2, \gamma >0\) are given.
The analytic solution is
\begin{gather}
  u(x,t)=c_1 + B_1 \mathrm{erf}\left( \frac{x}{\sqrt{4 k_1 t}}\right)
  \\
  \xi(t) = \alpha \sqrt{t}
  \intertext{where}
  \mathrm{erf}(x)=\frac{2}{\sqrt{\pi}} \int_0^x e^{-z^2} \,dz,
  \\
  B_1 =-\frac{c_1}{\mathrm{erf}\left(\alpha/\sqrt{4 k_1}\right)},\quad
  \intertext{and \(\alpha{}\) is a solution of the transcendental equation}
  \frac{\sqrt{k_1} c_1 e^{-\frac{\alpha^2}{4 k_1}}}{ \mathrm{erf}\left(\alpha/\sqrt{4 k_1}\right)}
  = -\frac{\gamma \sqrt{\pi}}{2} \alpha
\end{gather}

We calculate
\begin{gather}
  u_x(x,t) = \frac{B_1}{\sqrt{\pi k_1 t}} \exp\left(-\frac{x^2}{4 k_1 t}\right)
  \intertext{so}
  k_1 u_x(s(t),t)
  %= B_1 \frac{k_1}{\sqrt{\pi k_1 t}} \exp\left(- \frac{\alpha^2 t}{4 k_1 t}\right)
  %\\
  = B_1 \frac{k_1}{\sqrt{\pi k_1 t}} \exp\left(- \frac{\alpha^2}{4 k_1}\right)
  \\
  = \frac{c_1 \sqrt{k_1} \exp\left(-\frac{\alpha^2}{4 k_1}\right)}{\mathrm{erf}\left(\alpha/\sqrt{4 k_1}\right)}
  \left(\frac{-1}{\sqrt{\pi t}}\right)
  \intertext{By virtue of $\alpha$ satisfying the above transcendental equation and $\dD{\xi}{t} \equiv \frac{\alpha}{2 \sqrt{t}}$, it follows that}
  u_x(s(t),t)
  = \gamma \frac{\sqrt{\pi}}{2} \alpha \left(\frac{1}{\sqrt{\pi t}}\right)
  \\
  = \gamma \frac{\alpha }{2 \sqrt{t}}
  =  \gamma \dD{\xi}{t}
\end{gather}
That is, the Neumann boundary condition is satisfied at $x=s(t)$.
Similarly,
\begin{gather*}
  k_1 u_x(0,t) = \frac{ k_1 B_1 }{\sqrt{\pi k_1 t}} = \frac{\sqrt{k_1} B_1}{\sqrt{\pi t}}
\end{gather*}
is the condition on the fixed boundary.

Note that the domain degenerates at $t=0$ (and the problem data becomes singular), which is not supported by our theoretical framework; therefore, the data is shifted in time by a fixed amount \verb+tShift+.
The \verb+fzero+ rootfinder is used to solve the necessary transcendental equation to find $\alpha$ (denoted by \verb+boundary_constant+ in the code.)
This solution is implemented in \verb+true_solution.m+, included in Section~\ref{sec:code-listing-true-solution}.
The test code in \verb+test_true_solution.m+, included in Section~\ref{sec:code-listing-test-true-solution}, simply verified that the output shape and size of the parameters is correct.

%\subsection*{Acknowledgement}

\appendix
\section{Code Listings}
{\small 
\subsection{Forward}\label{sec:code-listing-forward}
\lstinputlisting[style=MATLAB-editor]{Forward.m}

\subsection{test\_Forward}\label{sec:code-listing-test-forward}
\lstinputlisting[style=MATLAB-editor]{test_Forward.m}

\subsection{Adjoint}\label{sec:code-listing-adjoint}
\lstinputlisting[style=MATLAB-editor]{Adjoint.m}

\subsection{test\_Adjoint}\label{sec:code-listing-test-adjoint}
\lstinputlisting[style=MATLAB-editor]{test_Adjoint.m}

\subsection{grad\_a}\label{sec:code-listing-grad-s}
\lstinputlisting[style=MATLAB-editor]{grad_s.m}

\subsection{true\_solution}\label{sec:code-listing-true-solution}
\lstinputlisting[style=MATLAB-editor]{true_solution.m}

\subsection{test\_true\_solution}\label{sec:code-listing-test-true-solution}
\lstinputlisting[style=MATLAB-editor]{test_true_solution.m}
}

\section{Change of Variables Details}
The order with which variables are referenced has a meaning inside the adjoint and forward codes.
In particular, the change of variables being taken, and when it is taken, affects the output, in general; in this small note we document the change of variables used to transform those problems to a rectangular domain, since it is implemented multiple times in the code and has led to issues before.
It is essential for the convergence of the adjoint code that the following transformation be taken \emph{before} reversing the problem in time.
Define the change of variables
\begin{gather*}
  (x,t) \mapsto (y,\bar{t}),
  \\
  \bar{t} = t,\quad
  y = x/s(t)
  \intertext{Define a new function}
  \tilde{\psi}(y,\bar{t}) = \psi(y s(\bar{t}), \bar{t})
\end{gather*}
If we have computed values of $\psi$ in $\Omega$, we may compute
\begin{gather*}
  \tilde{\psi}_y(y,\bar{t})
  = \D{}{y}\psi(y s(\bar{t}), \bar{t})
  = \psi_x(y s(\bar{t}), \bar{t}) s(\bar{t})
  \\
  \tilde{\psi}_{yy}(y,\bar{t}) = \psi_{xx}(y s(\bar{t}), \bar{t})
  \\
  \tilde{\psi}_{\bar{t}}
  = \D{}{\bar{t}} \psi(y s(\bar{t}), \bar{t})
  = \psi_x(ys(\bar{t}), \bar{t}) s'(\bar{t}) + \psi_t(ys(\bar{t}), \bar{t})
\end{gather*}

The function $\psi$ is defined by the inverse transformation,
\begin{gather*}
  (y,\bar{t}) \mapsto (x,t),
  \\
  t = \bar{t},\quad
  x = y s(\bar{t})
  \intertext{so}
  \psi(x,t) = \tilde{\psi}\left(x/s(t), t\right)
\end{gather*}
Suppose we have computed values $\tilde{\psi}(y,\bar{t})$ in the domain $[0,1]\times[0,T]$ and would like to compute those of $\psi$; we have
\begin{gather*}
  \psi_x(x,t) = \D{}{x} \tilde{\psi}\left(x/s(t), t\right)
  = \tilde{\psi}_y\left( x/s(t), t\right) / s(t)
  \intertext{so in particular}
  \psi_x(s(t), t) = \tilde{\psi}_y(1,t)/s(t)
  \\
  \psi_{xx}(x,t) = \tilde{\psi}_{yy}\left(x/s(t), t\right) / s^2(t)
  \\
  \psi_t(x,t) = -xs'(t)/s^2(t)\tilde{\psi}_y\left(x/s(t), t\right) + \tilde{\psi}_{\bar{t}}(x/s(t), t)
\end{gather*}

\section{Model Problem Zoo}
\subsection{Separable Solution}

The model problem currently implemented in \verb+test_Forward.m+, included in Section~\ref{sec:code-listing-test-forward} is~\eqref{eq:reduced-pde}--\eqref{eq:reduced-stefan} with a separable analytic solution and identically zero heat sources.
That is, consider

\begin{gather}
  Lu \equiv a(t) u_{xx} - u_{t} = f,~\text{in}~\Omega\nonumber
  \\
  u(x,0) = \phi (x),~0 \leq x \leq s(0)=s_0\nonumber
  \\
  a(t) u_x (0,t) = g(t),~0 \leq t \leq T\nonumber
  \\
  a(t) u_x (s(t),t)
  +  s'(t)
  = 0,~0 \leq t \leq T\nonumber
  \end{gather}
  where the analytic solution is given by
  \begin{gather}
  a(t) = a_{\text{true}}(t) \equiv 1,
  \\
  u(x,t) = T(t) X(x)
  \intertext{so}
  u_x(x,t) = T(t) X'(x)
  \intertext{and}
  u_{xx}(x,t) = T(t) X''(x)
  \\
  u_t = T'(t) X(x)
\end{gather}
Choosing $T(0)=1$ and asserting further that $T'=T$, we see that to satisfy~\eqref{eq:reduced-pde} with right-hand side identically zero, we must have
\[
  u_{xx} - u_t = T X'' - T' X = T (X'' - X) = 0
\]
Here, we see that we may take any solution of $X''-X=0$ satisfying a given boundary condition at $x=0$, with price that the ODE that $s(t)$ must solve may be complicated.
For instance, we may take the condition $u_x(0,t)=0$, which implies
\[
  X(x) = \cosh(x)
\]
is a solution, and hence
\begin{gather}
  \phi(x) = \cosh(x),
  \\
  u(x,t) = \exp(t) \cosh(x)
  \intertext{and the analytic function $g(t)$ is given by}
  g(t) = a(t) u_x(0,t) = \exp(t) \sinh(0) = 0
  \intertext{To find the function $x=s(t)$, we solve the ODE~\eqref{eq:reduced-stefan}, which takes the form}
  0 = T(t) X'(s(t)) + s'(t)
  = \exp(t)\sinh(s(t)) + s'(t),
  \intertext{Choosing the normalization}
  s(0)=1=:s_0 % Can be any normalizing condition here
  \intertext{we can check that the analytic solution is}
  s(t) \equiv 2 \arccoth\left(\exp(\exp(t)-1) \coth(1/2)\right)
\end{gather}


\bibliographystyle{siam}
\bibliography{refs}

\end{document}