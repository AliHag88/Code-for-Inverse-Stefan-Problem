\documentclass[letterpaper, 10pt]{amsart}

\usepackage[T1]{fontenc}
\usepackage{graphicx, baskervald}
\usepackage{todonotes}
\usepackage{hyperref}
\usepackage{geometry}

\usepackage{matlab-prettifier}

\usepackage[inline]{showlabels}

\let\institute\address%
\newtheorem{theorem}{Theorem}
\newtheorem{lemma}{Lemma}
\newtheorem{corollary}{Corollary}
\theoremstyle{definition}
\newtheorem{definition}{Definition}
\theoremstyle{remark}
\newtheorem{remark}{Remark}

\newcommand{\D}[2]{\frac{\partial{} #1}{\partial{} #2}}
\newcommand{\bk}[1]{\left\{#1\right\}}
\newcommand{\norm}[1]{\left\Vert #1\right\Vert}%chktex 1 ok (math-mode)
\newcommand{\lnorm}[1]{\left\vert #1\right\vert}%chktex 1 ok (math-mode)

\begin{document}

\title{Implementation of ISP}
\author{Ali Haqverdiyev \and Jonathan Goldfarb}
\institute{Department of Mathematical Sciences\\Florida Institute of
    Technology\\Melbourne, FL 32901}
\subjclass[2010]{TBD}
\date{\today}

\begin{abstract}
    TBD

    \smallskip%
    \noindent\textbf{\keywordsname.} TBD
\end{abstract}
\maketitle

\tableofcontents{}

\section{Introduction and Statement of Main Results}\label{sec:introduction}

\subsection{Inverse Stefan Problem}\label{sec:inverse-stefan-problem}
Consider the general one-phase Stefan problem:
\begin{gather}
    Lu \equiv {(a(t) b(x) u_x)}_x - u_{t} = f,~\text{in}~\Omega \label{eq:intro-pde}
    \\
    u(x,0) = \phi (x),~0 \leq x \leq s(0)=s_0 \label{eq:intro-init}
    \\
    a(t) b(0) u_x (0,t) = g(t),~0 \leq t \leq T \label{eq:pde-bound}
    \\
    a(t) b(s(t)) u_x (s(t),t) + \gamma (s(t),t) s'(t) = \chi (s(t),t),~0 \leq t \leq
    T \label{eq:pde-stefan}
    \\
    u(s(t),t) = \mu (t),~0 \leq t \leq T, \label{eq:pde-freebound}
    \intertext{where}
    \Omega = \{(x,t): 0 < x < s(t),~0 < t \leq T\}\label{eq:pde-domain-defn}
\end{gather}
where $a$, $b$, $f$, $\phi$, $g$, $\gamma$, $\chi$, $\mu$ are given functions.
Assume now that $a$ is unknown; in order to find $a$, along with $u$ and $s$, we must have additional information.
Assume that we are able to measure the temperature on our domain and the position of the free boundary at the final moment $T$.
\begin{equation}
    u(x,T) = w(x),~ 0 \leq x \leq s(T) = s_*.\label{eq:pde-finaltemp}
\end{equation}
Under these conditions, we are required to solve an \emph{inverse} Stefan problem (ISP): find a tuple
\def\controlvarsWithArgs{s(t), a(t)}
\def\controlvars{s, a}
\def\controlvarsWithDelta{{\Delta s}, {\Delta a}}
\[
    \bk{
        u(x,t), \controlvarsWithArgs
    }
\]
that satisfy conditions~\eqref{eq:intro-pde}--\eqref{eq:pde-finaltemp}.
ISP is not well posed in the sense of Hadamard: the solution may not exist; if it exists, it may not be unique, and in general it does not exhibit continuous dependence on the data.
The main methods available for ISP are based on a variational formulation, Frechet differentiability, and iterative gradient methods.
We cite recent papers~\cite{abdulla13,abdulla15} and the monograph~\cite{goldman97} for a list of references.
The established variational methods in earlier works fail in general to address two issues:
\begin{itemize}
    \item The solution of ISP does not depend continuously on the phase transition temperature $\mu(t)$ from~\eqref{eq:pde-freebound}.
    A small perturbation of the phase transition temperature may imply significant change of the solution to the inverse Stefan problem.
    \item In the existing formulation, at each step of the iterative method a Stefan problem must be solved (that is, for instance, the unknown heat flux $g$ is given, and the corresponding $u$ and $s$ are calculated) which incurs a high computational cost.
\end{itemize}

A new method developed in~\cite{abdulla13,abdulla15} addresses both issues with a new variational formulation.
The key insight is that the free boundary problem has a similar nature to an inverse problem, so putting them into the same framework gives a conceptually clear formulation of the problem; proving existence, approximation, and differentiability is a resulting challenge.
Existence of the optimal control and the convergence of the sequence of discrete
optimal control problems to the continuous optimal control problem was proved
in~\cite{abdulla13,abdulla15}.
In~\cite{abdulla16,abdulla17}, Frechet differentiability of the new variational
formulation was developed, and a full result showing Frechet differentiability
and the form of the Frechet differential
with respect to the free boundary, sources, and coefficients were proven.
Our goal in this work is to document the implementation in MATLAB using the built-in PDEPE solver.


\subsection{Forward Problem for ISP}\label{sec:forward-problem}
We will consider the problem~\eqref{eq:intro-pde}--\eqref{eq:pde-stefan} where $(\controlvars{}) \in V_R$ is fixed, and
\begin{align}
  V_R & =\Big\{
        v=(\controlvars{}) \in H, \lnorm{v}_H \leq R;
        ~s(0) = s_0,
        ~g(0) = a(0)b(0) \phi'(0),
        ~a_0 \leq a(t)\nonumber
  \\
      & \qquad
        ~\chi(s_{0},0) = \phi'(s_{0})a(0)b(s_0) + \gamma(s_0, 0)s'(0),
        ~0 < \delta \leq s(t)
        \Big\},\label{eq:control-set}
\end{align}
where $\beta_0, \beta_1, \beta_2 \geq 0$ and $a_0, \delta, R > 0$ are given, and
\def\acontrolspace{W_2^1}
\def\scontrolspace{W_2^2}
\begin{gather*}
  H := \acontrolspace(0,T)
  \times \scontrolspace(0,T)
    \\
    \norm{v}_H := \max\Big(
    \norm{a}_{\acontrolspace(0,T)},
    \norm{s}_{\scontrolspace(0,T)}
    \Big)
  \end{gather*}
  Define
  \let\l\ell%
\[
  D := \bk{(x,t) : 0\leq x\leq \l,~ 0\leq t\leq T},
\]
where $l = l(R) > 0$ is chosen such that for any control $v\in V_R$, its component
$s$ satisfies $s(t)\leq l$.
For a given control vector $v = (\controlvars{}) \in V_R$ transform the domain $\Omega$ to the cylindrical domain $Q_T$
by the change of variables $y = x / s(t)$.
Let $d = d(x, t)$, $(x, t) \in \Omega$ stand for any of $u$, $b$, $f$, $\gamma$, $\chi$, define the function $\tilde{d}$ by
\begin{gather*}
  \tilde{d}(x,t) = d\big(x s(t), t\big),~
  \tilde{b}(x) = b(x s(t)),~\text{and}~
  \tilde{\phi}(x) = \phi\big( x s_0\big)
\end{gather*}
\def\utilde{\tilde{u}}
The transformed function $\utilde$ is a \emph{pointwise a.e.} solution of the Neumann problem
\begin{gather}
  \frac{a}{s^2}\big(\tilde{b} \utilde_y\big)_y + \frac{y s'}{s} \utilde_y - \utilde_{t} = \tilde{f}, ~\text{in}~Q_T\label{eq:tform-pde}
  \\
  \utilde(x,0) = \tilde{\phi}(x), ~0 \leq x \leq 1 \label{eq:tform-iv}
  \\
  a(t) b(0) \utilde_y(0, t) = g(t)s(t), ~0 \leq t \leq T   \label{eq:tform-lbdy}
  \\
  a(t) b(1) \utilde_y(1, t) = \tilde{\chi}(1, t) s(t) - \tilde{\gamma}(1,t)s'(t)s(t), ~0 \leq t \leq T \label{eq:tform-rbdy}
\end{gather}


\subsection{MATLAB Implementation of Forward Problem}
The built-in MATLAB PDE solver, \verb+pdepe+, solves parabolic-elliptic problems in one space dimension.
In particular, it uses a method-of-lines technique as a differential-algebraic equation solver applied to problems of the form
\begin{gather}
  c\left(x,t,u,\D{u}{x}\right) \D{u}{t}
  = x^{-m} \D{}{x} \left( x^m F\left(x,t,u,\D{u}{x} \right)\right)
  + d\left( x,t,u,\D{u}{x}\right)\label{eq:matlab-pde}
  \intertext{for}
  a \leq x \leq b,\quad t_0 \leq t \leq t_f
  \intertext{where $m=0,1,2$ is fixed, $c$ is a diagonal matrix of size $n\times n$, where there are $n$ components in $u$.
    There must be at least one parabolic equation, which corresponds to the condition of at least one component of $c$ being positive.
    The solution components satisfy}
  u(x,t_0) = u_0(x)\label{eq:matlab-ic}
  \intertext{and boundary conditions of the form}
  p(x,t,u)
  + q(x,t) F\left(x,t,u,\D{u}{x} \right)\Big\vert_{x=a,b}
  = 0\label{eq:matlab-bc}
\end{gather}
In particular, the function $F$ appearing in the boundary condition is the same as the flux term in the PDE.
In the MATLAB implementation, the method \verb+Forward+ is intended to produce a function $u(x,t)$ by computing $\utilde(y,t)$ and returning $u(x,t) = \utilde(x/s(t),t)$.
The subfunction \verb+pdeSolver+ handles the conversion of problems of the form~\eqref{eq:tform-pde}--\eqref{eq:tform-rbdy} to the form that the MATLAB solver requires.

We first write system~\eqref{eq:tform-pde}--\eqref{eq:tform-rbdy} in the form
\begin{gather}
  % a \big(\tilde{b} \utilde_y\big)_y
  % + s y s' \utilde_y
  % - s^2\utilde_{t}
  % = s^2 \tilde{f}, ~\text{in}~Q_T
  % \\
  % \intertext{Or, multiplying the first equation through by $-1$,}
  s^2 \utilde_t = a \big(\tilde{b} \utilde_y\big)_y
  + s y s' \utilde_y
  - s^2 \tilde{f}, ~\text{in}~Q_T=:(0,1) \times (0,T)
  \\
  \utilde(x,0) = \tilde{\phi}(x), ~0 \leq x \leq 1
  \\
  a(t) b(0) \utilde_y(0, t) = g(t)s(t), ~0 \leq t \leq T
  \\
  a(t) \tilde{b}(1) \utilde_y(1, t)
  = \tilde{\chi}(1, t) s(t)
  - \tilde{\gamma}(1,t)s'(t)s(t), ~0 \leq t \leq T
\end{gather}

Evidently, we must take $a=0$, $b=1$, $t_0=0$, $t_f = T$, and $m=0$; % , giving
% \[
% c\left(x,t,u,\D{u}{x}\right) \D{u}{t} = \D{}{x} \left( F\left(x,t,u,\D{u}{x} \right)\right) + d\left( x,t,u,\D{u}{x}\right)
% \]
% By
then, identifying the diffusion coefficient between the two problems, we observe that
\[
  F(x, t, u, p) := a(t) \tilde{b}(x) p
\]
and by considering the time derivative, evidently
\[
  c(x,t,u,p) := s^2(t)
\]
and lastly,
\[
  d(x,t,u,p) := s(t) \big(s'(t) x p - s(t) \tilde{f}(x,t)\big)
\]
The initial data should be taken as $u_0(x) = \tilde{\phi}(x) = \phi(x s(t))$, and the boundary conditions take the form
\begin{gather*}
  a(t)\tilde{b}(0) \utilde_y(0,t) \equiv F(0, t, \utilde, \utilde_y)
  = p(0,t,\utilde):=g(t) s(t),
  \\
  q(0,t)=1
  \\
  a(t)\tilde{b}(1) \utilde_y(1,t) \equiv F(1, t, \utilde, \utilde_y)
  = p(1,t,\utilde)
  := s(t)\left(
    \tilde{\chi}(1,t) - \tilde{\gamma}(1,t) s'(t)
    \right)
\end{gather*}

\subsection{Formulation for Reduced Model}
Consider the problem~\eqref{eq:tform-pde}--\eqref{eq:tform-rbdy} with $\gamma\equiv 1$, $\chi \equiv 0$, and $b \equiv 1$.
Then the equations above simplify to give
\begin{gather}
  % a \big(\tilde{b} \utilde_y\big)_y
  % + s y s' \utilde_y
  % - s^2\utilde_{t}
  % = s^2 \tilde{f}, ~\text{in}~Q_T
  % \\
  % \intertext{Or, multiplying the first equation through by $-1$,}
  s^2 \utilde_t = a \big( \utilde_y\big)_y
  + s y s' \utilde_y
  - s^2 \tilde{f}, ~\text{in}~Q_T
  \\
  \utilde(y,0) = \tilde{\phi}(y), ~0 \leq x \leq 1
  \\
  a(t) \utilde_y(0, t) = g(t)s(t), ~0 \leq t \leq T
  \\
  a(t) \utilde_y(1, t)
  = - s'(t)s(t), ~0 \leq t \leq T
\end{gather}
and the mapping from the notation in our paper to the notation preferred by MATLAB becomes
\begin{gather*}
  F(x, t, u, p) := a(t) p
  \\
  c(x,t,u,p) := s^2(t)
  \\
  d(x,t,u,p) := s(t) \big(s'(t) x p - s(t)\tilde{f}(x,t)\big)
\intertext{The initial data should be taken as $u_0(x) = \tilde{\phi}(x) = \phi(x s(t))$, and the boundary conditions take the form}
  a(t) \utilde_y(0,t) \equiv F(0, t, \utilde, \utilde_y)
  = p(0,t,\utilde)
  :=g(t) s(t),
  \\
  a(t) \utilde_y(1,t) \equiv F(1, t, \utilde, \utilde_y)
  = p(1,t,\utilde)
  := -s(t) s'(t)
  \intertext{and}
  q(x,t) \equiv 1
\end{gather*}

Under the additional assumption that the sources $f \equiv 0$, the mapping simplifies further to
\begin{gather*}
  F(x, t, u, p) := a(t) p
  \\
  c(x,t,u,p) := s^2(t)
  \\
  d(x,t,u,p) := s(t) s'(t) x p
\end{gather*}
The boundary conditions take the same form with the updated definition of $F$.
This is the model implemented in \verb+Forward.m+, included in section~\ref{sec:code-listing-forward}.

\subsection{Optimal Control Problem and Gradient-Based Approach}\label{sec:statement-of-main-results}
Consider the minimization of the functional
\def\J{\mathcal{J}}
\begin{equation}
  \J(v)
  = \beta_0 \int_0^{s(T)} \lnorm{u(x, T; v) - w(x)}^2\,dx
  + \beta_1 \int_0^T \lnorm{u(s(t), t; v) - \mu(t)}^2\,dt
  + \beta_2 \lnorm{s(T) - s_*}^2\label{eq:functional}
\end{equation}
on the control set~\eqref{eq:control-set}.
\def\I{\mathcal{I}}
The formulated optimal control problem~\eqref{eq:functional},~\eqref{eq:control-set},~\eqref{eq:tform-pde}--\eqref{eq:tform-rbdy} will be called Problem $\I$.

% The main results are established under the assumptions
% \begin{gather}
%     0 < a_0 \leq a(x,t)~\text{in}~D\label{eq:a-ellipticity-cond}
%     \\
%     a,b_x,b \in \coeffspace(D)\label{eq:datacond-coeff}
%     \\
%     w \in \ivspace(0,\l),\quad \phi \in \ivspace(0, s_{0}),\label{eq:datacond-iv}
%     \\
%     \chi,\gamma \in \chigammaspace(D)\label{eq:datacond-4}
%     \\
%     \mu \in \muspace[0,T]\label{eq:datacond-mu}
% \end{gather}
% Given a control vector $v \in V_R$, under the conditions~\eqref{eq:a-ellipticity-cond}--\eqref{eq:datacond-mu} there exists a unique pointwise a.e.\ solution $u \in W_2^{2,1}(\Omega)$ of the Neumann problem~\eqref{eq:intro-pde}--\eqref{eq:pde-stefan} such that the transformed function $\utilde(y,t) = u(y s(t),t)$ belongs to $\solnspace(Q_T)$ and solves~\eqref{eq:tform-pde}--\eqref{eq:tform-rbdy} (\cite{solonnikov64}; see Lemma~\ref{lem:j-well-defined} below.)
\begin{definition}\label{defn:adjoint}
  For given $v$ and $u = u(x, t; v)$, $\psi$ is a solution to the adjoint problem if
  \begin{gather}
    L^{*} \psi := \big(a b \psi_x\big)_x + \psi_t = 0,\quad\text{in}~\Omega \label{eq:adj-pde}
    \\
    \psi(x, T) = 2\beta_0(u(x, T) - w(x)),~0 \leq x \leq s(T) \label{eq:adj-finalmoment}
    \\
    b(0)a(t)\psi_x(0, t) =0,~0 \leq t \leq T \label{eq:adj-robin-fixed}
    \\
    \Big[a b \psi_x - s'\psi - 2\beta_1(u - \mu)\Big]_{x=s(t)}=0, ~0 \leq t \leq T \label{eq:adj-robin-free}
  \end{gather}
\end{definition}


\begin{theorem}\label{thm:gradient-result}
  Problem $\I$ has a solution, and the functional $\J(v)$ is differentiable in the sense of Frechet, and the first variation is
  % \begin{gather}
  %   \left\langle \J'(v),\Delta v \right\rangle_H
  %   = \int_0^T \Big[2\beta_1\big(u - \mu\big)u_x + \psi \big(\chi_x - \gamma_x s'  - a (b u_x)_x \big)\Big]_{x=s(t)} {\Delta s}(t)\,dt \nonumber
  %   \\
  %   +
  %   \left[\beta_0\lnorm{u(s(T),T) - w(s(T))}^2 + 2 \beta_2 (s(T) - s_*)\right] {\Delta s}(T) \nonumber
  %   \\
  %   + \int_0^T {\Delta a} \int_0^{s(t)} {(b u_x)}_x \psi \,dx \,dt
  %   - \int_0^T \big[\psi \gamma\big]_{x=s(t)} {\Delta s}'(t) \,dt \nonumber
  %   \\
  %   - \int_0^T {\Delta a} \left[ [ b u_x\psi]_{x=s(t)} + [b u_x \psi]_{x=0}\right] \,dt\label{eq:gradient-full}
  % \end{gather}
  \begin{gather}
    \left\langle \J'(v),\Delta v \right\rangle_H
    = \int_0^T \Big[2\beta_1\big(u - \mu\big)u_x + \psi \big(\chi_x - \gamma_x s'  - a (b u_x)_x \big)\Big]_{x=s(t)} {\Delta s}(t)\,dt \nonumber
    \\
    +
    \left[\beta_0\lnorm{u(s(T),T) - w(s(T))}^2 + 2 \beta_2 (s(T) - s_*)\right] {\Delta s}(T) 
    - \int_0^T \big[\psi \gamma\big]_{x=s(t)} {\Delta s}'(t) \,dt \nonumber
    \\
    +\int_0^T {\Delta a} \left[ \int_0^{s(t)} {(b u_x)}_x \psi \,dx - [b u_x\psi]_{x=s(t)} - [b u_x \psi]_{x=0}\right] \,dt\label{eq:gradient-full}
  \end{gather}
  where $\J'(v) \in H'$ is the Frechet derivative, $\langle \cdot,\cdot \rangle_H$ is a pairing between $H$ and its dual $H'$, $\psi$ is a solution to the adjoint problem in the sense of definition~\ref{defn:adjoint}, and $\delta v = (\controlvarsWithDelta{})$ is a variation of the control vector $v \in V_R$ such that $v + \delta v \in V_R$.
\end{theorem}

\subsection{MATLAB Implementation of Adjoint Problem}

To implement the adjoint problem, we first change variable as $\bar{t} = T-t$ to derive
\begin{gather}
  L^{*} \psi := \big(a(T-\bar{t}) b(x) \psi_x\big)_x - \psi_{\bar{t}} = 0,\quad\text{in}~\Omega
  \\
  \psi(x, T) = 2\beta_0(u(x, T) - w(x)),~0 \leq x \leq s(T)
  \\
  b(0)a(\bar{t}-T)\psi_x(0, \bar{t}) =0,~0 \leq \bar{t} \leq T
  \\
  a(T-\bar{t}) b(s(T-\bar{t})) \psi_x(x,\bar{t})
  - s'(T-\bar{t})\psi(x,\bar{t})
  \\
  - 2\beta_1\big(
  u(s(T-\bar{t}),T-\bar{t}) - \mu(T-\bar{t})\big)=0, ~0 \leq t \leq T
\end{gather}

\subsection{Model Problem \#1}
As a model problem, we will take~\eqref{eq:intro-pde}--\eqref{eq:pde-stefan} with $b\equiv 1$, $\chi \equiv 0$, and $\gamma \equiv 1$:
\begin{gather}
  Lu \equiv {(a(t) u_x)}_x - u_{t} = f,~\text{in}~\Omega
  \\
  u(x,0) = \phi (x),~0 \leq x \leq s(0) = s_0
  \\
  a(t) u_x (0,t) = g(t),~0 \leq t \leq T
  \\
  a(t) u_x (s(t),t) + s'(t) = 0,~0 \leq t \leq
  T
\end{gather}
In this case, the adjoint problem~\eqref{eq:adj-pde}--\eqref{eq:adj-robin-free} then takes the form
\begin{gather}
  L^{*} \psi := \big(a \psi_x\big)_x + \psi_t = 0,\quad\text{in}~\Omega
  \\
  \psi(x, T) = 2\beta_0(u(x, T) - w(x)),~0 \leq x \leq s(T)
  \\
  a(t)\psi_x(0, t) =0,~0 \leq t \leq T
  \\
  \Big[a \psi_x - s'\psi - 2\beta_1(u - \mu)\Big]_{x=s(t)}=0, ~0 \leq t \leq T
\end{gather}
and the gradient is
\begin{gather}
  \left\langle \J'(v),\Delta v \right\rangle_H
  = \int_0^T \Big[2\beta_1\big(u - \mu\big)u_x - a \psi u_{xx} \Big]_{x=s(t)} {\Delta s}(t)\,dt \nonumber
  \\
  +
  \left[\beta_0\lnorm{u(s(T),T) - w(s(T))}^2 + 2 \beta_2 (s(T) - s_*)\right] {\Delta s}(T)
  - \int_0^T \psi \big\vert_{x=s(t)} {\Delta s}'(t) \,dt\nonumber
  \\
  + \int_0^T {\Delta a} \left[
    \int_0^{s(t)}  u_{xx} \psi \,dx
    - [u_x \psi]_{x=0}
    - [ u_x\psi]_{x=s(t)}
  \right]\,dt
\end{gather}

%\subsection*{Acknowledgement}

\appendix
\section{Code Listings}
\subsection{Forward}\label{sec:code-listing-forward}
\lstinputlisting[style=MATLAB-editor]{Forward.m}

\bibliographystyle{siam}
\bibliography{refs}

\end{document}